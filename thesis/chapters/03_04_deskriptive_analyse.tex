\subsection{Deskriptive Analyse}

Um die Komplexität des Datensatzes und des Optimierungsproblems zu quantifizieren, wurde eine deskriptive Analyse durchgeführt.

Der vollständige Datensatz besteht aus insgesamt 381 \textit{Offerings} ($|\mathcal{O}|$) und 28 \textit{Courses}, woraus sich eine durchschnittliche Anzahl von 13.6 \textit{Offerings} pro \textit{Course} ergibt.

% Feldman und Golumbic erzeugten für ihre Experimente zufällige Ausgangsdaten mit 10-20 Planpunkten, die durchschnittlich 2.5 \textit{Offerings} pro Planpunkt hatten \cite{feldmangolumbic}.


Ein wesentlicher Faktor für die Schwierigkeit der Stundenplanerstellung ist die zeitliche Dichte der angebotenen Kurse. Tabelle \ref{descriptive_analysis_offering_weekday} zeigt die Verteilung der Lehrveranstaltungen über die Wochentage.


\begin{table}[h]
    \begin{tabular}{l|r}
        \hline
        \textbf{Wochentag} & \textbf{Anteil (\%)} \\ \hline
        Montag             & 21.1\%               \\
        Dienstag           & 25.4\%               \\
        Mittwoche          & 21.8\%               \\
        Donnerstag         & 19.7\%               \\
        Freitag            & 11.3\%               \\
        Samstag            & 0.7\%                \\
        Sonntag            & 0\%                  \\ \hline
        \textbf{Total}     & \textbf{100.0\%}     \\ \hline
    \end{tabular}
    \caption{Prozentuale Verteilung der Kurse auf Wochentage}
    \label{descriptive_analysis_offering_weekday}
\end{table}

Wie aus den Daten ersichtlich wird, konzentriert sich der Großteil des Lehrangebots (ca. 88 \%) auf die Kernzeit von Montag bis Donnerstag, während der Freitag mit 11.3 \% bereits deutlich seltener belegt ist. Samstage spielen im regulären Lehrbetrieb nahezu keine Rolle. Diese ungleiche Verteilung verschärft die Problematik zeitlicher Überschneidungen ($\mathcal{F}_{Zeit}$) an Tagen mit besonders vielen Kursen.



\newpage



Zusätzlich zur Verteilung auf die Tage ist die zeitliche Platzierung innerhalb eines Tages entscheidend für die Modellierung von Präferenzen. Tabelle \ref{descriptive_analysis_start_hour_of_offerings} schlüsselt die Startzeiten der einzelnen Kurseinheiten auf.

\begin{table}[h]
    \begin{tabular}{l|r}
        \hline
        \textbf{Stunde} & \textbf{Anteil (\%)} \\ \hline
        7:00            & 0.6\%                \\
        8:00            & 16.9\%               \\
        9:00            & 8.3\%                \\
        10:00           & 10.1\%               \\
        11:00           & 6.1\%                \\
        12:00           & 8.0\%                \\
        13:00           & 11.4\%               \\
        14:00           & 8.1\%                \\
        15:00           & 5.9\%                \\
        16:00           & 10.3\%               \\
        17:00           & 5.6\%                \\
        18:00           & 8.8\%                \\
        19:00           & 0.0\%                \\ \hline
    \end{tabular}
    \caption{Prozentuale Verteilung der Start-Uhrzeit der Kurse}
    \label{descriptive_analysis_start_hour_of_offerings}
\end{table}

Die Analyse der Startzeiten zeigt, dass die WU Wien ein sehr breites Zeitfenster nutzt. Ein signifikanter Anteil der Kurse beginnt bereits um 08:00 Uhr (16.9 \%), was für Studierende mit Präferenzen für einen späteren Vorlesungsbeginn (vgl. Szenario \ref{scenario_free_day}) eine hohe Anzahl an potenziell zu vermeidenden Einheiten bedeutet.



\newpage



Neben dem Beginn der Einheiten variiert auch die Dauer der einzelnen Termine, was die Berechnung von Überschneidungsfreiheiten komplexer gestaltet als bei starren Zeitslots. Tabelle \ref{descriptive_analysis_offering_duration} stellt die Dauer der Einheiten dar. Nicht ganzstündige Einheiten wurden kaufmännisch gerundet.


\begin{table}[h]
    \begin{tabular}{l|r}
        \hline
        \textbf{Stunden} & \textbf{Anteil (\%)} \\ \hline
        1                & 0.7\%                \\
        2                & 27.1\%               \\
        3                & 41.9\%               \\
        4                & 27.2\%               \\
        5                & 1.8\%                \\
        6                & 0.8\%                \\
        7                & 0.1\%                \\
        8                & 0.3\%                \\ \hline
    \end{tabular}
    \caption{Prozentuale Verteilung der Dauer der Kurse}
    \label{descriptive_analysis_offering_duration}
\end{table}


Die überwiegende Mehrheit der Veranstaltungen (über 96 \%) dauert zwischen zwei und vier Stunden. Da die Termine jedoch individuell als Zeitstempel (Start/Ende) vorliegen und nicht in ein fixes Raster gepresst sind, muss der Algorithmus die Überschneidung der Zeitintervalle berechnen, anstatt Slots zu vergleichen.



% \begin{figure}[h]
%     \centering
%     \begin{tikzpicture}
%         \begin{axis}[
%                 ybar,                          % Specifies a bar chart
%                 ymin=0, ymax=20,               % Y-axis range
%                 width=\textwidth,              % Adjust width to fit page
%                 height=7cm,
%                 bar width=0.6cm,               % Width of the individual bars
%                 ylabel={Anteil (\%)},          % Y-axis label
%                 xlabel={Stunde},               % X-axis label
%                 symbolic x coords={7:00, 8:00, 9:00, 10:00, 11:00, 12:00, 13:00, 14:00, 15:00, 16:00, 17:00, 18:00, 19:00},
%                 xtick=data,                    % Use the labels provided in coordinates
%                 nodes near coords,             % Displays the value above the bar
%                 nodes near coords style={font=\tiny, /pgf/number format/fixed},
%                 every node near coord/.append style={rotate=90, anchor=west}, % Rotates labels if they overlap
%                 axis x line*=bottom,
%                 axis y line*=left,
%                 enlarge x limits=0.05,
%             ]
%             \addplot[fill=blue!40, draw=blue] coordinates {
%                     (7:00, 0.6) (8:00, 16.9) (9:00, 8.3) (10:00, 10.1)
%                     (11:00, 6.1) (12:00, 8.0) (13:00, 11.4) (14:00, 8.1)
%                     (15:00, 5.9) (16:00, 10.3) (17:00, 5.6) (18:00, 8.8) (19:00, 0.0)
%                 };
%         \end{axis}
%     \end{tikzpicture}
%     \caption{Histogramm der prozentualen Verteilung der Kurse nach Uhrzeit}
% \end{figure}
