\subsection{Implementierung des Offering-Order Algorithmus}

Basierend auf der Methodik von Feldman und Golumbic \cite{feldmangolumbic} wurde der Optimierungsalgorithmus \textit{Offering Order} implementiert, der auf einer heuristischen Sortierung basiert.


\subsubsection{Sortierung\label{offordersorting}}

Der Algorithmus nutzt die berechneten \textit{Marks}, um eine Suchreihenfolge festzulegen:

\begin{enumerate}
    \item Alle \textit{Offerings} innerhalb des selben Planpunkts werden absteigend nach \textit{Mark} sortiert. Innerhalb eines Planpunkts wird somit das Offering mit dem höchsten \textit{Mark} zuerst in Betracht gezogen.
    \item Die Planpunkte selbst werden nach dem \textit{Mark} des besten \textit{Offering} innerhalb des Planpunkts sortiert.
\end{enumerate}


\subsubsection{Suchstrategie}

Der optimale Stundenplan wird durch eine \textit{Forward-Checking Backtracking} Strategie gesucht. Es wird folgendermaßen vorgegangen:

\begin{enumerate}
    \item Die Planpunkte werden iterativ durchlaufen. Gestartet wird mit dem Planpunkt, der das \textit{Offering} mit der höchsten \textit{Mark} beinhaltet (siehe \ref{offordersorting}).
    \item Für jeden Planpunkt wird versucht, das \textit{Offering} mit der höchsten \textit{Mark} auszuwählen. Bevor die Auswahl getroffen wird, wird geprüft, ob das Hinzufügen des \textit{Offerings} zu dem aktuellen Teilstundenplan zu einer zeitlichen Überschneidung führt oder andere Constraints verletzt (\textbf{Forward-Checking})
    \item Führt das Hinzufügen eines \textit{Offerings} zu keinem gültigen Endergebnis in den nachfolgenden Planpunkten, wird die Auswahl rückgängig gemacht und das nächstbeste \textit{Offering} der aktuellen Gruppe wird getestet (\textbf{Backtracking})
\end{enumerate}

