\section{Einleitung}


Kaum eine Entscheidung im Alltag von Studierenden erscheint so simpel und erweist sich gleichzeitig so komplex wie die Erstellung des eigenen Stundenplans. Was auf den ersten Blick nach einer organisatorischen Routineaufgabe aussieht, entpuppt sich bei näherer Betrachtung als vielschichtiges Optimierungsproblem: Überschneidungen zwischen Lehrveranstaltungen, begrenzte Kursplätze und individuelle Präferenzen wie Arbeitszeiten und Lerngewohnheiten machen die Planung zu einer Herausforderung

In einer Zeit, in der viele Entscheidungsprozesse bereits durch Algorithmen unterstützt werden, stellt sich die Frage, ob auch die Stundenplanerstellung automatisiert und optimiert werden kann. Hier setzt die folgende Arbeit an: Sie untersucht, wie sich das \textit{Student Schedulung Problem (SSP)} mithilfe von verschiedenen Optimierungsverfahren modellieren und lösen lässt.

% TODO: schreibe so etwas wie "das Problem besteht schon lange. bereits im jahre 1990 haben sich feldmangolumbic damit auseinander gesetzt". wie gut funktionieren diese algorithmen unter den heutigen technischen gegebenheiten und für das Vorlesungsverzeichnis der WU Wien

