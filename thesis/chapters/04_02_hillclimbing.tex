\subsection{Implementierung der Hill-Climbing Algorithmen}

Basierend auf der Methodik von Feldman und Golumbic \cite{feldmangolumbic} wurden die Optimierungsalgorithmen \textit{Hill Climbing v1} und \textit{Hill Climbing v3} implementiert. Das Ziel dieser Algorithmen ist es, inkrementell, durch lokal optimale Entscheidungen einen Stundenplan zu finden, der die Zielfunktion (\textit{Mark}) maximiert.





\subsubsection{Hill Climbing v1}

Der \textit{Hill Climbing v1}-Algorithmus wählt bei jedem Schritt das beste verfügbare \textit{Offering} aus \textbf{allen verfügbaren Offerings} gemessen am \textit{Mark} des Stundenplans inklusive des gewählten \textit{Offering} \cite{gh-hc1}.





\subsubsection{Hill Climbing v3}

Der \textit{Hill Climbing v3}-Algorithmus unterscheidet sich vom \textit{v1}-Ansatz dadurch, dass der Suchraum für einen neuen Kurs reduziert wird. Anstatt alle verfügbaren \textit{Offerings} zu evaluieren, beschränkt \textit{v3} die Auswahl auf eine Teilmenge der nach \textit{Mark} sortierten Liste.
Konkret wird in der Implementierung nur die zweite Hälfte --- die Offerings mit der individuell besten \textit{Mark} --- der verfügbaren \textit{Offerings} betrachtet \cite{gh-hc3}.





\subsubsection{Suchstrategie}

Ausgehend vom aktuellen Stundenplan wird für jeden möglichen nächsten Zustand die \textit{Mark} berechnet, indem der Stundenplan ergänzt um ein weiteres \textit{Offering} durch die Evaluationsfunktion bewertet wird. Anschließend wird jenes Lehrangebot ausgewählt, das zu der höchsten Bewertung führt, und in den Stundenplan übernommen.

\begin{equation}
    \label{eq:v1_selection}
    a^* = \arg\max_{a \in \text{available\_offerings}} \left( \text{get\_schedule\_mark}(\text{schedule} \cup \{a\}) \right)
\end{equation}


Nur jene \textit{Offerings} werden in die Auswahl aufgenommen, die keine Nebenbedingungen verletzen (zeitliche Überschneidung, Planpunkt bereits erfüllt, \textit{Offering} bereits gewählt, \textit{Offering} in Liste verbotener Kurse). Die Schleife wird fortgesetzt, solange ein gültiges \textit{Offering} gefunden wird, das zu einer Verbesserung oder Erweiterung des Stundenplans führt.





\subsubsection{Abbruchkriterien}
Der Algorithmus bricht ab, wenn einer der folgenden Zustände eintritt:
\begin{enumerate}
    \item Der Stundenplan ist valide und hat die maximale Anzahl von Lehrveranstaltungen erreicht.
    \item Der Stundenplan ist valide, und es kann keine weitere gültige \textit{Offering} gefunden werden, ohne dass eine Nebenbedingung verletzt wird.
    \item Es kann keine weitere \textit{Offering} gefunden werden, die keine Constraints verletzt. Wenn die minimale Anzahl der Lehrveranstaltungen noch nicht erfüllt ist, führt das zu einem ungültigen Ergebnis.
\end{enumerate}


\begin{algorithm}[H]
    \caption{Hill Climbing}
    \begin{algorithmic}[1]

        \State $\mathit{schedule} \gets \emptyset$
        \State $\mathit{availableOfferings} \gets \textsc{FilterConflicts}(\mathit{offerings})$

        \While{$\mathit{availableOfferings} \neq \emptyset$}

        \State $\mathit{nextOffering} \gets \arg\max_{o \in \mathit{availableOfferings}} \, \textsc{Mark}(\mathit{schedule} \cup \{o\})$

        \If{\textsc{IsValid}$(\mathit{schedule} \cup \{\mathit{nextOffering}\})$}
        \State $\mathit{schedule} \gets \mathit{schedule} \cup \{\mathit{nextOffering}\}$
        \EndIf

        \If{$|\mathit{schedule}| = \textsc{MaxScheduleLength}$}
        \State \Return $\mathit{schedule}$
        \EndIf

        \State $\mathit{availableOfferings} \gets \textsc{FilterConflicts}(\mathit{offerings})$

        \EndWhile

        \State \Return $\mathit{schedule}$

    \end{algorithmic}
\end{algorithm}





\subsubsection{Zur Nicht-Implementierung von Hill Climbing v2}

Der von Feldman und Golumbic vorgeschlagene \textit{Hill Climbing v2}-Algorithmus priorisiert zusätzlich auf der Ebene von \textit{Gruppen} (\textit{Groups}). Im Originalkontext des Papers beziehen sich Gruppen auf die Unterscheidung zwischen Pflichtveranstaltungen (\textit{Mandatory}) und Wahlveranstaltungen (\textit{Elective}), wobei die Priorität der Gruppen die Auswahl von Pflichtveranstaltungen vor Wahlveranstaltungen erzwingt \cite{feldmangolumbic}.
Da es im Hauptstudium Wirtschaftsinformatik keine Unterscheidung zwischen Pflicht- und Wahlveranstaltungen gibt, wäre die Implementierung funktional identisch mit \textit{v1}.