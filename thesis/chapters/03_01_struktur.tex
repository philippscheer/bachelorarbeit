\subsection{Struktur des Datensatzes}

Der Datensatz des Vorlesungsverzeichnisses der Wirtschaftsuniversität Wien für das Sommersemester 2025 bildet die Grundlage für die vorliegende Arbeit. Die Daten wurden automatisiert unter Verwendung des beschriebenen Prozesses \ref{alg:datenextraktion} extrahiert.

Die Modellierung des Studiums an der WU basiert auf zwei zentralen Entitäten, die für das Verständnis der Optimierungsalgorithmen wichtig sind:

\begin{itemize}
    \item \textbf{Course (Planpunkt):} Ein \textit{Course} repräsentiert ein zu absolvierendes Modul, das mit einer festen ECTS-Anzahl assoziiert ist. Aus den verfügbaren \textit{Offerings} eines Planpunkts darf maximal eines ausgewählt werden.
    \item \textbf{Offering (Lehrveranstaltung): } Ein \textit{Offering} ist eine Lehrveranstaltung, die einem Planpunkt angehört. Jedes Offering besitzt eine eindeutige ID (von $1 - 9999$) und ist mit konkreten Termin- und Zeitangaben hinterlegt, die für die Prüfung von zeitlichen Konflikten verwendet wird.
\end{itemize}
